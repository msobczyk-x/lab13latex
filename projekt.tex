\documentclass[11pt,a4paper]{article}
\usepackage[utf8]{inputenc}
\usepackage{graphicx}
\usepackage{tabularx}
\usepackage[cp1250]{inputenc}
\usepackage{latexsym}
\usepackage{polski}

\frenchspacing
\begin{document}
\begin{titlepage}
    \begin{center}
        \vspace*{1cm}
            
        \Huge
        \textbf{Rozwój komputerów}
            
        \vspace{0.5cm}
        \LARGE
        Jak komputery rozwijały się na przestrzeni lat.
            
        \vspace{1.5cm}
            
        \textbf{Maciej Sobczyk}
            
        \vfill
            
        
            
        \vspace{2cm}
            
        \includegraphics[width=0.6\textwidth]{komputer.png}
        \vspace{3cm}
            
        \Large
        Nr albumu 275515 \\
        Uniwersytet Gdański\\
        \Date {24 Styczeń 2021}
            
    \end{center}
\end{titlepage}
\tableofcontents
\newpage
\section{Definicja komputera}
Komputer (od ang. computer); dawniej: mózg elektronowy, elektroniczna maszyna cyfrowa, maszyna matematyczna – maszyna przeznaczona do przetwarzania informacji, które da się zapisać w formie ciągu cyfr albo sygnału ciągłego. Maszyna roku tygodnika „Time” w 1982 roku.


\section{Pierwsze komputery}
Pierwszym w historii komputerem był ENIAC. Było to już ponad 70 lat temu! Komputer został zaprojektowany w 1945 roku przez naukowców z Uniwersytetu w Pensylwanii. Publicznie zaprezentowano go już rok później na Uniwersytecie w Princeton. Jak nietrudno zgadnąć, ENIAC nie miał jakichś szczególnie, jak na dzisiejsze standardy wysokich osiągów.\\ Pracował z taktowaniem zaledwie 0,1 MHz, to w tamtych czasach robiło olbrzymie wrażenie. Dla porównania nawet proste współcześnie stosowane smartfony są około 50 tysięcy razy szybsze! Do wykonania ENIACA konieczne było zastosowanie ponad 70 tysięcy rezystorów, 1500 przekaźników, 10 tysięcy kondensatorów oraz 6 tysięcy ręcznych przełączników. Do sprawnej obsługi ENIACA potrzebny był też cały sztab ludzi. Sam komputer zajmował też bardzo dużo miejsca, bo aż 167 m2 powierzchni. Miał 2,4 m wysokości i aż 24 metry długości. Ważył również niemało, bo aż 27 ton.\\ Co ciekawe stworzono go z myślą o ułatwieniu produkcji tablic balistycznych. Jego koszt sięgnął około 6 milionów dolarów.

\begin{figure}
\centering
\caption{Inżynier obsługujący ENIAC}
   \includegraphics[width=0.8\textwidth]{eniac.jpg} 
   
\end{figure}
\subsection{Procesor - "głowa komputera"}
Sekwencyjne urządzenie cyfrowe, które pobiera dane z pamięci operacyjnej, interpretuje je i wykonuje jako rozkazy. Procesory wykonują ciągi prostych operacji matematyczno-logicznych ze zbioru operacji podstawowych.\\

Procesory wykonywane są zwykle jako układy scalone zamknięte w hermetycznej obudowie, często posiadającej złocone wyprowadzenia (stosowane ze względu na odporność na utlenianie) i w takiej postaci nazywa się je mikroprocesorami – w mowie potocznej pojęcia procesor i mikroprocesor używane są zamiennie. Sercem procesora jest monokryształ krzemu, na który naniesiono techniką fotolitografii szereg warstw półprzewodnikowych, tworzących, w zależności od zastosowania, sieć od kilku tysięcy do kilku miliardów tranzystorów. Jego obwody wykonywane są z metali o dobrym przewodnictwie elektrycznym, takich jak aluminium czy miedź.\\

Jedną z podstawowych cech procesora jest określona długość (liczba bitów) słowa, na którym wykonuje on podstawowe operacje obliczeniowe. Jeśli przykładowo słowo tworzą 64 bity, to taki procesor określany jest jako 64-bitowy. Innym ważnym parametrem określającym procesor jest szybkość, z jaką wykonuje on rozkazy. Przy danej architekturze procesora, szybkość ta w znacznym stopniu zależy od czasu trwania pojedynczego taktu[1], a więc głównie od częstotliwości jego taktowania.\\
\subsection{Porównanie mocy obliczeniowej}

\begin{table}[h!]
\centering
    \begin{tabularx}{0.8\textwidth} { 
  | >{\raggedright\arraybackslash}X 
  | >{\centering\arraybackslash}X 
  | >{\raggedleft\arraybackslash}X | }
 \hline
 Nazwa & Rok & Moc obliczeniowa \\
 \hline
 ENIAC  & 1945  & 0,1 MHz  \\

\hline
 CDC6600  & 1964  & 10 MHz  \\

\hline
 Cray-2  & 1985 & 244 MHz  \\

\hline
 Commodore 64  & 1982  & 1 MHz  \\
\hline
\end{tabularx}
\caption{Tabela porównująca moce obliczeniowe pierwszych komputerów}
\label{table:1}
\end{table}

\section{Typy komputerów}
Współcześnie komputery dzieli się na:
\begin{itemize}
    \item komputery osobiste („PC”, z ang. personal computer) – o rozmiarach umożliwiających ich umieszczenie na biurku, używane zazwyczaj przez pojedyncze osoby
    \item komputery domowe – poprzedniki komputerów osobistych, korzystające z telewizora jako monitora
    \item konsola – komputer wyspecjalizowany w programach rozrywkowych. Zazwyczaj korzysta z telewizora jako głównego wyświetlacza
    \item komputery mainframe – często o większych rozmiarach, których zastosowaniem jest przetwarzanie dużych ilości danych na potrzeby różnego rodzaju instytucji, pełnienie roli serwerów itp.
    \item superkomputery – największe komputery o dużej mocy obliczeniowej, używane do czasochłonnych obliczeń naukowych i symulacji skomplikowanych systemów.
    \item komputery wbudowane – (lub osadzone, ang. embedded) specjalizowane komputery służące do sterowania urządzeniami z gatunku automatyki przemysłowej, elektroniki użytkowej (np. telefony komórkowe itp.) czy wręcz poszczególnymi komponentami wchodzącymi w skład komputerów
\end{itemize}
\begin{figure}
\centering
\caption{Superkomputer - IBM Summit}
\includegraphics[width=0.7\textwidth]{summit.jpg} 
   
\end{figure}



\newpage
\listoftables
\newpage
\begin{thebibliography}{5}
\bibitem{komputer-wikipedia}
 Computer (ang.). 
\textit{The Online Etymology Dictionary} 
\texttt{http://www.etymonline.com/index.php?term=computer}

\bibitem{Historia}
 Historia komputerów - kiedy wynaleziono pierwszy komputer
\texttt{https://progres24.pl/blog/historia-komputerow-kiedy\\-wynaleziono-pierwszy-komputer}

\bibitem{benchmark}
Pierwszy komputer obchodził 75. urodziny - zobacz jak rozwijały się pecety 

\texttt{https://www.benchmark.pl/aktualnosci/historia-rozwoju-\\komputerow-i-laptopow.html}

\bibitem{purepc}
 Krótka historia komputerów osobistych - 70 lat rozwoju PC 
\texttt{https://www.purepc.pl/krotka-historia-komputerow-osobistych\\-70-lat-rozwoju-pc}
\end{thebibliography}
\end{document}